\documentclass{beamer}
%
% Choose how your presentation looks.
%
% For more themes, color themes and font themes, see:
% http://deic.uab.es/~iblanes/beamer_gallery/index_by_theme.html
%
\mode<presentation>
{
  \usetheme{Madrid}      % or try Darmstadt, Madrid, Warsaw, ...
  \usecolortheme{default} % or try albatross, beaver, crane, ...
  \usefonttheme{default}  % or try serif, structurebold, ...
  \setbeamertemplate{navigation symbols}{}
  \setbeamertemplate{caption}[numbered]
}

\usepackage[english]{babel}
\usepackage[utf8x]{inputenc}
\usepackage{graphicx}
\usepackage{array}

\title[05-Análise Léxica]{EA879 -- Introdução ao Software Básico\\Análise Léxica}
\author{Tiago F. Tavares}
\institute{FEEC -- UNICAMP}
\date{Aula 05 -- 15/agosto/2017}

\begin{document}

\begin{frame}
  \titlepage
\end{frame}

% Uncomment these lines for an automatically generated outline.
%\begin{frame}{Outline}
%  \tableofcontents
%\end{frame}

\section{Introdução}

\begin{frame}{Objetivos}
  \Large
  \begin{itemize}
    \item Entender o que é e como usar o Lex.
    \item Entender a sintaxe do Lex.
    \item Entender como compilar programas usando Lex.
    \item Construir uma pequena aplicação usando Lex.
  \end{itemize}
\end{frame}

\begin{frame}[fragile]{Exercício}
  \centering
  No código abaixo, identifique os liteirais inteiros e os números em ponto
  flutuante:

  \begin{verbatim}
  a = 50;
  b = 50.5 + 20;
  c = 60.2;
  \end{verbatim}

\end{frame}

\begin{frame}{Exercício}
  Ordene as instruções abaixo de forma a permitir reconhecer floats e inteiros
  num código:
  \begin{enumerate}
    \item Aplicar expressão regular \textsc{[0-9]+}
    \item Aplicar expressão regular \textsc{[0-9]+[.][0-9]*}
    \item Aplicar expressão regular \textsc{[0-9]*}
    \item Caso sucesso, encontrei inteiro
    \item Caso sucesso, encontrei float
    \item Caso contrário...
  \end{enumerate}

\end{frame}

\begin{frame}{Live Coding}
  \centering
  \Large
  Lex é uma ferramenta que permite programar reconhecedores de tipos e criar
  programas que têm comportamentos distintos ao receber entradas de tipos
  diferentes.
\end{frame}

\begin{frame}{Exercício}
  \centering
  \large
  Proponha uma modificação ao programa feito ao vivo que permita somar todos os
  literais tipo int encontrados na string de entrada.
\end{frame}

\begin{frame}{Discussão}
\large
  Até o momento, neste curso, vimos uma série de conteúdo sobre expressões
  regulares. Junto ao seu grupo, faça um mapa conceitual sobre expressões
  regulares:
  \begin{enumerate}
    \item Escreva numa folha de papel (espalhe bem!) os \textbf{nomes} de todos
      os conceitos que fizeram parte da aula (por exemplo: \textit{expressão
      regular} e \textit{sintaxe}). Se quiser, inclua conceitos que não fizeram
      parte da aula mas que foram importantes.
    \item Após, ligue os nomes usando setas rotuladas por verbos, dando
      significado à sua lista de nomes (por exemplo: \textit{sintaxe}
      \textcolor{red}{permite programar} \textit{expressão regular}). Tente
      ligar todos os conceitos do seu mapa.
  \end{enumerate}

\end{frame}



\end{document}
