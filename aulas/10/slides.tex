\documentclass{beamer}
%
% Choose how your presentation looks.
%
% For more themes, color themes and font themes, see:
% http://deic.uab.es/~iblanes/beamer_gallery/index_by_theme.html
%
\mode<presentation>
{
  \usetheme{Madrid}      % or try Darmstadt, Madrid, Warsaw, ...
  \usecolortheme{default} % or try albatross, beaver, crane, ...
  \usefonttheme{default}  % or try serif, structurebold, ...
  \setbeamertemplate{navigation symbols}{}
  \setbeamertemplate{caption}[numbered]
}

\usepackage[english]{babel}
\usepackage[utf8x]{inputenc}
\usepackage{graphicx}
\usepackage{array}

\title[10-compartilhamento-de-recursos]{EA879 -- Introdução ao Software
Básico\\Políticas de Compartilhamento de Recursos}
\author{Tiago F. Tavares}
\institute{FEEC -- UNICAMP}
\date{Aula 10 -- 15/agosto/2017}

\begin{document}

\begin{frame}
  \titlepage
\end{frame}

% Uncomment these lines for an automatically generated outline.
%\begin{frame}{Outline}
%  \tableofcontents
%\end{frame}

\section{Introdução}

\begin{frame}{Objetivos}
  \Large
  \begin{itemize}
    \item Entender o que significam recursos compartilhados.
    \item Entender o impacto de políticas de administração de recursos.
    \item Refletir sobre quem deve administrar os recursos compartilhados.
  \end{itemize}
\end{frame}


\begin{frame}[fragile]{Problema: recursos compartilhados}
  \centering
  \large
  Na Fictícia Escola de Engenharia de Computação, há 100 alunos, mas apenas 1
  computador. Isso é um problema porque há uma disciplina que pede que alunos
  enviem, individualmente, programas de computador ao SUper SIstema. Junto ao
  seu grupo, proponha 3 formas diferentes de organizar o uso do computador,
  sabendo que todos os alunos têm o mesmo prazo e que alguns alunos demoram mais
  tempo que outros para fazer programas de computador.
\end{frame}

\begin{frame}[fragile]{First Come, First Serve (FCFS)}
  \centering
  \large
  O primeiro aluno que sentou no computador continua a usá-lo até terminar sua
  tarefa. Os seguintes fazem uma fila e executam o mesmo procedimento.
  \begin{enumerate}
  \item Qual fração do tempo é perdida fazendo login na máquina ou salvando
      o trabalho?
  \item Como seria possível ``quebrar'' esse sistema, dominando completamente o
    computador indefinidamente?

  \end{enumerate}
\end{frame}

\begin{frame}[fragile]{Round-Robin}
  \centering
  \large
  Os alunos que querem usar o computador fazem uma fila. Ao sentar-se no
  computador, um aluno pode usá-lo por 1 minuto, e depois deve retornar ao fim
  da fila, cedendo sua vez ao próximo.
  \begin{enumerate}
  \item Qual fração do tempo é perdida fazendo login na máquina ou salvando
      o trabalho?
  \item Como seria possível reduzir essa fração?
  \end{enumerate}
\end{frame}

\begin{frame}[fragile]{Prazos diferenciados}
  \centering
  \large
  Os alunos da Fictícia Escola de Engenharia de Computação se dividiram em duas
  turmas, A e B. Os alunos da turma A têm que entregar um código daqui a 24h. Os
  alunos da turma B têm que entregar um código na semana que vem. Proponha uma
  nova forma de organização, levando em consideração esse novo prazo.
\end{frame}

\begin{frame}[fragile]{Trabalhos diferenciados}
  \centering
  \large
  Os alunos da Fictícia Escola de Engenharia de Computação se dividiram em duas
  turmas, A e B. O prazo de entrega do trabalho de todos eles é o mesmo (daqui a
  uma semana), mas os alunos da turma A têm que entregar somente uma função que
  avalia expressões regulares, enquanto os alunos da turma B têm que entregar um
  webserver com sete níveis de segurança e interface ``touchscreen'' para iPhone
  e Android. Proponha uma nova forma de organização, levando em consideração os
  tamanhos diferenciados dos trabalhos.
\end{frame}

\begin{frame}[fragile]{Round-Robin com prioridades}
  \centering
  \large
  Cada aluno recebe um número chamado \textit{prioridade}. A cada bloco de tempo
  (15 minutos, por exemplo), o usuário do computador é trocado pelo próximo da
  fila. Porém, um aluno não pode assumir a máquina se outro aluno, com
  prioridade mais alta, também estiver na fila.
  \begin{enumerate}
    \item Como seriam formas efetivas de definir as prioridades?
    \item Como seria possível ``quebrar'' o sistema, evitando que outros alunos
      usem a máquina?
  \end{enumerate}
\end{frame}

\begin{frame}[fragile]{Round-Robin com prioridades e proporção}
  \centering
  \large
  Cada aluno recebe um número chamado \textit{prioridade}. O computador é
  revezado pelos alunos (round-robin) de forma que o bloco de tempo atribuído a
  cada aluno é proporcional à sua prioridade.
  \begin{enumerate}
  \item Como seriam formas efetivas de definir as prioridades?
  \item Como seria possível ``quebrar'' o sistema, evitando que outros alunos
    usem a máquina?
  \end{enumerate}
\end{frame}

\begin{frame}[fragile]{Sistemas de computação}
  \centering
  \Large
  No caso dos alunos da Fictícia Escola, qual é o \textit{recurso
  compartilhado}? E, quando executamos programas no computador, quais recursos
  são compartilhados? Em que situação dois programas ``disputam'' para usar o
  mesmo recurso?
\end{frame}

\begin{frame}[fragile]{Sistemas de computação}
  \centering
  \Large
  Através da sua observação, os programas que executam no seu celular operam em
  fila (first-come, first-serve) ou em rodízio (round-robin)? Por que você acha
  isso?
\end{frame}

\begin{frame}[fragile]{Sistemas de computação}
  \centering
  \Large
  O algoritmo kernel de tempo real que vimos na aula passada opera como fila
  (first-come, first-serve) ou em rodízio (round-robin)? Por que você acha isso?
  Seria possível mudar esse comportamento? Como?
\end{frame}


\begin{frame}[fragile]{Sistemas de computação}
  \centering
  \large
  Reflita junto ao seu grupo e decida quais alternativas são verdadeiras em
  relação ao revezamento do computador na Fictícia Escola:
  \begin{enumerate}
    \item Precisamos ``salvar'' o que cada aluno está fazendo para que ele
      possa retomar seu trabalho posteriormente.
    \item Cada um pode ser responsável por controlar seu próprio tempo de uso
      do computador.
    \item Uma equipe de administração deve ser responsável por controlar o tempo
      do computador.
    \item É desejável que cada aluno tenha uma ``área de trabalho''
      isolada dos demais.
    \item É desejável que existam áreas de disco que podem ser compartilhadas
      entre alunos.
    \item Para fazer seu trabalho, um aluno não precisa saber de nenhuma
      informação sobre o trabalho dos outros alunos, apenas de quando será sua
      vez de usar novamente o computador.
  \end{enumerate}

\end{frame}


\end{document}
