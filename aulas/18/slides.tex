\documentclass{beamer}
%
% Choose how your presentation looks.
%
% For more themes, color themes and font themes, see:
% http://deic.uab.es/~iblanes/beamer_gallery/index_by_theme.html
%
\mode<presentation>
{
  \usetheme{Madrid}      % or try Darmstadt, Madrid, Warsaw, ...
  \usecolortheme{default} % or try albatross, beaver, crane, ...
  \usefonttheme{default}  % or try serif, structurebold, ...
  \setbeamertemplate{navigation symbols}{}
  \setbeamertemplate{caption}[numbered]
}

\usepackage[english]{babel}
\usepackage[utf8x]{inputenc}
\usepackage{graphicx}
\usepackage{array}

\title[18-Pilha]{EA879 -- Introdução ao Software
Básico\\O Heap}
\author{Tiago F. Tavares}
\institute{FEEC -- UNICAMP}
\date{Aula 18 -- 7/novembro/2017}

\begin{document}

\begin{frame}
  \titlepage
\end{frame}

% Uncomment these lines for an automatically generated outline.
%\begin{frame}{Outline}
%  \tableofcontents
%\end{frame}

\section{Introdução}

\begin{frame}{Objetivos}
  \Large
  \begin{itemize}
    \item Entender como funciona alocação dinâmica de memória
    \item Entender como e memória pode ser ``devolvida'' ao sistema
  \end{itemize}
\end{frame}

\begin{frame}[fragile]{Previously, on EA879...}
  \centering
  \Large
  \begin{itemize}
    \item Compiladores: RegEx e GLC
    \item Processos, threads, preempção e mutexes
    \item Pilha de execução: resolve sintaxe!
  \end{itemize}
\end{frame}

\begin{frame}[fragile]{Exercício 1}
  \centering
  \Large
  Exercício 1
\end{frame}

\begin{frame}[fragile]{Exercício 2}
  \centering
  \Large
  Exercício 2
\end{frame}

\begin{frame}[fragile]{Heap}
  \centering
  \Large
  O Heap (não confundir com a estrutura de dados \textit{heap}) é uma região da
  memória que tem uma lista de todas as posições de memória ainda não usadas
  pelo programa. As funções \textsc{malloc()} e \textsc{free()} buscam por
  memória livre usando o Heap (que não é implementado usando a estrutura de
  dados \textit{heap}). A memória alocada usando o Heap não é devolvida
  ``automaticamente'' ao sistema após o retorno da função que alocou a memória.
\end{frame}

\begin{frame}[fragile]{Exercício 3}
  \centering
  \Large
  Exercício 3
\end{frame}

\begin{frame}[fragile]{Exercício 4}
  \centering
  \Large
  Exercício 4
\end{frame}

\begin{frame}[fragile]{Exercício 5}
  \centering
  \Large
  Exercício 5
  \begin{enumerate}
    \item <2-> O programa funcionaria, porque a memória alocada dinamicamente só
      é usada dentro da própria função e, portanto, pode ser devolvida ao
      sistema quando a função retorna.
    \item <3-> O programa do ex. 2 não funcionaria porque a memória alocada
      dentro da função \textsc{novo\_registro()} é usada depois,
      em outras funções, e, portanto, não
      pode ser devolvida ao sistema quando a função retorna.
  \end{enumerate}
\end{frame}

\begin{frame}[fragile]{Exercício 6}
  \centering
  \Large
  Exercício 6
\end{frame}

\begin{frame}[fragile]{Conclusão}
  \centering
  \Large
  Quando eu deveria usar \textsc{malloc}, \textsc{alloca} ou variáveis
  pré-definidas (automáticas)?
  \begin{enumerate}
    \item Não sei o tamanho do vetor que precisarei, mas só vou usá-lo no escopo
    local.
    \item Preciso inicializar um \textsc{struct} que será usado por todo o
    programa.
    \item Preciso inicializar um número, inicialmente indefinido,
    de \textsc{struct}s que serão usados por todo o programa.
    \item Vou inicializar um vetor em uma função, mas usá-lo em outra função.
    \item Sei exatamente o tamanho do vetor que vou precisar usar, e ele só será
    usado no escopo local.
  \end{enumerate}

\end{frame}



\end{document}
