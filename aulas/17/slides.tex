\documentclass{beamer}
%
% Choose how your presentation looks.
%
% For more themes, color themes and font themes, see:
% http://deic.uab.es/~iblanes/beamer_gallery/index_by_theme.html
%
\mode<presentation>
{
  \usetheme{Madrid}      % or try Darmstadt, Madrid, Warsaw, ...
  \usecolortheme{default} % or try albatross, beaver, crane, ...
  \usefonttheme{default}  % or try serif, structurebold, ...
  \setbeamertemplate{navigation symbols}{}
  \setbeamertemplate{caption}[numbered]
}

\usepackage[english]{babel}
\usepackage[utf8x]{inputenc}
\usepackage{graphicx}
\usepackage{array}

\title[17-Pilha]{EA879 -- Introdução ao Software
Básico\\A Pilha}
\author{Tiago F. Tavares}
\institute{FEEC -- UNICAMP}
\date{Aula 17 -- 31/outubro/2017}

\begin{document}

\begin{frame}
  \titlepage
\end{frame}

% Uncomment these lines for an automatically generated outline.
%\begin{frame}{Outline}
%  \tableofcontents
%\end{frame}

\section{Introdução}

\begin{frame}{Objetivos}
  \Large
  \begin{itemize}
    \item Lembrar como funciona uma pilha
    \item Entender como chamadas de função são implementadas
    \item Antever consequências de segurança dessa implementação
  \end{itemize}
\end{frame}

\begin{frame}[fragile]{Previously, on EA879...}
  \centering
  \Large
  \begin{itemize}
    \item Compiladores: RegEx e GLC
    \item Processos, threads, preempção e mutexes
    \item Deadlock!
  \end{itemize}
\end{frame}

\begin{frame}[fragile]{Exercício 1}
  \centering
  \Large
  \begin{enumerate}
  \item <2-> vazia
  \item <3-> 50
  \item <4-> vazia
  \item <5-> 30
  \item <6-> 20
  \item <7-> 30
  \item <8-> 60
  \end{enumerate}

\end{frame}

\begin{frame}[fragile]{Exercício 2}
  \centering
  \Large
  Exercício 2
\end{frame}

\begin{frame}[fragile]{Exercício 3}
  \centering
  \Large
  Exercício 3
\end{frame}

\begin{frame}[fragile]{Exercício 4}
  \centering
  \Large
  Exercício 4
\end{frame}

\begin{frame}[fragile]{Exercício 5}
  \centering
  \Large
  Antes do exercício 5, ouça atentamente à exposição sobre variáveis, escopo e
  pilha. Durante a exposição, tente fazer um desenho do processo que será
  descrito.
\end{frame}

\begin{frame}[fragile]{Exercício 6}
  \centering
  \Large
  Exercício 6
\end{frame}

\begin{frame}[fragile]{Exercício 7}
  \centering
  \Large
  Exercício 7
\end{frame}

\begin{frame}[fragile]{Exercício 8}
  \centering
  \Large
  Exercício 8
  \begin{enumerate}
    \item <2-> 1 chamada recursiva = 1 push na pilha
    \item <3-> Se a memória for infinita: infinitas chamadas
    \item <4-> Memória RAM
    \item <5-> Stack Overflow, Stack Smashing, Segmentation Fault
  \end{enumerate}
\end{frame}

\begin{frame}[fragile]{Conclusão}
  \centering
  \LARGE
  Bom feriado!!!
\end{frame}



\end{document}
