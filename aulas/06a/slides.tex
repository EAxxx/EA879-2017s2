\documentclass{beamer}
%
% Choose how your presentation looks.
%
% For more themes, color themes and font themes, see:
% http://deic.uab.es/~iblanes/beamer_gallery/index_by_theme.html
%
\mode<presentation>
{
  \usetheme{Madrid}      % or try Darmstadt, Madrid, Warsaw, ...
  \usecolortheme{default} % or try albatross, beaver, crane, ...
  \usefonttheme{default}  % or try serif, structurebold, ...
  \setbeamertemplate{navigation symbols}{}
  \setbeamertemplate{caption}[numbered]
}

\usepackage[english]{babel}
\usepackage[utf8x]{inputenc}
\usepackage{graphicx}
\usepackage{array}

\title[06a-Gramáticas]{EA879 -- Introdução ao Software Básico\\Gramáticas e
Aplicações}
\author{Tiago F. Tavares}
\institute{FEEC -- UNICAMP}
\date{Aula 06-A -- 28/agosto/2017}

\begin{document}

\begin{frame}
  \titlepage
\end{frame}

% Uncomment these lines for an automatically generated outline.
%\begin{frame}{Outline}
%  \tableofcontents
%\end{frame}

\section{Introdução}

\begin{frame}{Objetivos}
  \Large
  \begin{itemize}
    \item Entender problemas que podem ser abordados usando gramáticas
    \item Entender quais propriedades de gramáticas são relevantes para cada
      problema
  \end{itemize}
\end{frame}

\begin{frame}{Problemas com Gramáticas Regulares}
  \large
  Como é possível modelar os seguintes problemas usando gramáticas regulares:
  \begin{enumerate}
    \item Gerar uma string que corresponde a um número inteiro
    \item Detectar se uma string corresponde a um número inteiro
  \end{enumerate}
\end{frame}

\begin{frame}{Problemas com Gramáticas Livres de Contexto}
  \large
  Como é possível modelar os seguintes problemas usando gramáticas livres de
  contexto:
  \begin{enumerate}
    \item Gerar uma expressão matemática com parêntese balanceados
    \item Detectar se uma expressão matemática tem parênteses balanceados
  \end{enumerate}
\end{frame}

\begin{frame}{Inteligência Computacional: tipos de modelos}
  \large
  Abordagem gerativa (generative model)
  \begin{enumerate}
    \item Encontrar um modelo capaz de gerar o fenômeno que quero detectar
    \item Verificar se minha medição corresponde ao modelo
  \end{enumerate}

  Abordagem discriminativa (discriminative model)
  \begin{enumerate}
    \item Encontrar um modelo capaz de separar os fenômenos que quero
      classificar
    \item Verificar o resultado de aplicar o modelo sobre a minha medição
  \end{enumerate}

  Qual abordagem corresponde à de usar gramáticas?
\end{frame}

\begin{frame}{Criando modelos}
\large
  Quais dos problemas abaixo podem ser modelados usando gramáticas regulares ou
  gramáticas livres de contexto? Como cada modelo pode ser usado? Quais são suas
  limitações?
  \begin{enumerate}
    \item Gerar CPFs válidos
    \item Gerar sequências de notas musicais (representadas por números
      inteiros, onde 1 = dó e 7 = si, e sem considerar durações)
    \item Gerar uma expressão matemática
    \item Gerar conjugações de verbos à partir de um radical
    \item Gerar estruturas (sequências de versos e refrões) de músicas
      populares
    \item Gerar expressões matemáticas que expandem um resultado específico
    \item Gerar frases válidas em Português
  \end{enumerate}
\end{frame}

\begin{frame}{Tipos de problemas}
\large
  Quais tipos de problemas abaixo indicam o uso de gramáticas regulares? E quais
  poderiam ser resolvidos usando gramáticas livres de contexto?
  \begin{enumerate}
    \item Gerar sequências de tokens, cada um sem relação com o token anterior.
    \item Gerar sequências de tokens cujo agrupamento tem significado.
    \item Separar sílabas de uma palavra em portugês.
    \item Verificar se uma frase tem sujeito e verbo.
    \item Verificar se uma frase musical (sequência de notas) só tem notas que
      correpondem a uma escala.
    \item Compor uma estrutura musical adequada para ser tocada pela bateria da
      LEU.
  \end{enumerate}
\end{frame}

\begin{frame}{Problemas gerais}
\large
  Proponha um modelo, usando gramáticas, para cada um dos problemas abaixo (não
  precisa implementar, somente explicar como o modelo deve ser usado)
  \begin{enumerate}
    \item Partindo de um vocabulário de jogadas elementares de futebol (chutar,
      passar, correr, etc.), construir uma tática capaz de ganhar um jogo.
    \item Transformar frases escrita em ordem direta de tal forma que a saída
      soe como se fosse falada pelo Mestre Yoda.
    \item Verificar se uma frase tem concordância verbal em relação a número.
  \end{enumerate}
\end{frame}

\begin{frame}{Problemas avançados}
\large
  Proponha um modelo, usando gramáticas, para cada um dos problemas abaixo (não
  precisa implementar, somente explicar como o modelo deve ser usado)
  \begin{enumerate}
    \item Reconhecer o significado de uma frase.
    \item Gerar uma frase com significado conhecido.
    \item Traduzir uma frase de uma língua para outra, mantendo o significado.
  \end{enumerate}
\end{frame}

\begin{frame}{E o que mais?}
\large
Junto ao seu grupo, proponha um problema que pode ser modelado usando gramáticas
livres de contexto. Como o modelo deve ser usado? Por que ele é adequado ao seu
problema?
\end{frame}

\begin{frame}{Referências}
\large
  Para saber mais sobre alguns dos problemas que conversamos:
  \begin{enumerate}
\item X. Xu and H. Man, "Interpreting sports tactic based on latent context-free
  grammar," 2015 IEEE International Conference on Image Processing (ICIP),
      Quebec City, QC, 2015, pp. 4072-4076.
\item E. Nakamura, M. Hamanaka, K. Hirata and K. Yoshii, "Tree-structured
  probabilistic model of monophonic written music based on the generative theory
      of tonal music," 2016 IEEE International Conference on Acoustics, Speech
      and Signal Processing (ICASSP), Shanghai, 2016, pp. 276-280.
 \item P. Bahadur, A. Jain and D. S. Chauhan, "Architecture of English to
   Sanskrit machine translation," 2015 SAI Intelligent Systems Conference
      (IntelliSys), London, 2015, pp. 616-624.
  \end{enumerate}
\end{frame}






\end{document}
