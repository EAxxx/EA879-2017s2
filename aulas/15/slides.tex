\documentclass{beamer}
%
% Choose how your presentation looks.
%
% For more themes, color themes and font themes, see:
% http://deic.uab.es/~iblanes/beamer_gallery/index_by_theme.html
%
\mode<presentation>
{
  \usetheme{Madrid}      % or try Darmstadt, Madrid, Warsaw, ...
  \usecolortheme{default} % or try albatross, beaver, crane, ...
  \usefonttheme{default}  % or try serif, structurebold, ...
  \setbeamertemplate{navigation symbols}{}
  \setbeamertemplate{caption}[numbered]
}

\usepackage[english]{babel}
\usepackage[utf8x]{inputenc}
\usepackage{graphicx}
\usepackage{array}

\title[15-threads]{EA879 -- Introdução ao Software
Básico\\Threads}
\author{Tiago F. Tavares}
\institute{FEEC -- UNICAMP}
\date{Aula 15 -- 19/outubro/2018}

\begin{document}

\begin{frame}
  \titlepage
\end{frame}

% Uncomment these lines for an automatically generated outline.
%\begin{frame}{Outline}
%  \tableofcontents
%\end{frame}

\section{Introdução}

\begin{frame}{Objetivos}
  \Large
  \begin{itemize}
    \item Entender as semelhanças e diferenças entre threads e processos
    \item Aplicar análise de código como ferramenta de aprendizado
    \item Entender sincronização de threads (join)
  \end{itemize}
\end{frame}

\begin{frame}[fragile]{Previously, on EA879...}
  \centering
  \Large
  \begin{itemize}
  \item Processos
  \item Pipes
  \item Preempção
  \item Fork
  \item Memória compartilhada
  \item Dispatchers
  \item Produtor-consumidor
  \end{itemize}
\end{frame}

\begin{frame}[fragile]{Revisão}
  \centering
  \Large
  \begin{itemize}
    \item Seria correto dizer que threads são subdivisões de processos?

    \item <2-> Threads são linhas de execução que acontecem dentro de um processo. Então,
podemos dizer que processos são compostos de uma ou mais threads, e, portanto,
  threads são subdivisões de processos.
  \end{itemize}
\end{frame}

\begin{frame}[fragile]{Revisão}
  \centering
  \Large
  \begin{itemize}
    \item O texto da Wikipedia diz que uma vantagem de threads é que elas
      permitem simplificar modelos de computação. Proponha uma situação em que a
      solução usando threads é significativamente mais simples que a solução sem
      threads.

    \item <2-> Situações em que duas coisas acontecem de forma concorrente, mas
      que podem interromper uma à outra. Um webserver, por exemplo (há a
      tarefa de receber requisições e a tarefa de responder requisições).
      Também: programas com interface responsiva, programas que acessam várias
      URLs ao mesmo tempo...
  \end{itemize}
\end{frame}


\begin{frame}[fragile]{Revisão}
  \centering
  \Large
  \begin{itemize}
    \item Também segundo a Wikipedia, threads são ``mais rápidas de criar que
      processos porque não possuem quaisquer recursos associados a eles''. A
      quais recursos de máquina isso se refere?

    \item <2-> A memória é compartilhada entre threads, então não é preciso
      trocar a memória quando há uma troca da execução entre threads.
  \end{itemize}
\end{frame}

\begin{frame}[fragile]{Revisão}
  \centering
  \Large
  \begin{itemize}
    \item A terceira vantagem de threads, segundo a Wikipedia, é que, quando há
      uma grande quantidade de entradas e saídas, programas feitos em
      multi-thread são mais rápidos. Por que isso acontece? Proponha uma
      situação que exemplifique essa situação.

    \item <2-> Quando há um processo de E/S, a thread que realiza a E/S passa
      algum tempo parada, esperando uma resposta do dispositivo externo (por
      exemplo: esperando que um dado seja gravado no disco). Durante esse tempo,
      uma outra thread pode executar. Exemplo: acessar bancos de dados com
      muitas requisições.
  \end{itemize}
\end{frame}


\begin{frame}[fragile]{Exercício}
  \centering
  \Large
  Exercício 1
  \begin{enumerate}
    \item <2-> \textsc{pthread.h}
    \item <3-> \textsc{pthreat\_t}
    \item <4-> \textsc{pthread\_create()}
    \item <5-> \textsc{10}
  \end{enumerate}
\end{frame}

\begin{frame}[fragile]{Exercício}
  \centering
  \Large
  Exercício 2
  \begin{enumerate}
    \item <2-> \textsc{int M} tem escopo local
  \end{enumerate}
\end{frame}


\begin{frame}[fragile]{Exercício}
  \centering
  \Large
  Exercício 3
  \begin{enumerate}
    \item <2-> \textsc{pthread\_join(x)} espera a thread x terminar
    \item <3-> A thread principal só encerra depois que as threads criadas por
      ela terminam.
  \end{enumerate}
\end{frame}

\begin{frame}[fragile]{Exercício}
  \centering
  \Large
  Exercício 4
  \begin{enumerate}
    \item <2-> \textsc{produtor()} povoa um buffer com 20 inteiros de 0 a 19
    \item <3-> \textsc{consumidor()} escreve o buffer na tela
    \item <4-> Saída: escrever na tela números inteiros de 0 a 19.
  \end{enumerate}
\end{frame}

\begin{frame}[fragile]{Exercício}
  \centering
  \Large
  Exercício 5
  \begin{enumerate}
    \item <2-> \textsc{int total\_buffer} e suas referências
    \item <3-> Evitam que o consumidor leia números que não foram produzidos
      e evita que o produtor sobrescreva números que não foram lidos.
  \end{enumerate}
\end{frame}

\begin{frame}[fragile]{Mutex: eu tenho a chave!}
  \centering
  \Large
  Problema do banheiro do posto na estrada:
  a chave tem que ser retirada com o frentista.
  \begin{enumerate}
    \item Que situações isso evita?
    \item O que acontece se alguém pedir a chave, mas ela estiver emprestada?
    \item O que acontece se alguém não devolver a chave?
    \item <2->\textsc{mutex\_lock()} é ``pedir a chave''
    \item <2->\textsc{mutex\_unlock()} é ``devolver a chave''
    \item <2->Comportamento de espera é automático
    \item <2->O ``banheiro'' é chamado de \textbf{região crítica}.
  \end{enumerate}
\end{frame}

\begin{frame}[fragile]{Exercício}
  \centering
  \Large
  Exercício 6
  \begin{enumerate}
    \item <2-> As partes que se referem à alteração do número de elementos no
      buffer
    \item <3-> Evitam que outra thread acesse aquela região
    \item <4-> Por que o acesso concorrente a essa região pode gerar
      inconsistências de dados
  \end{enumerate}
\end{frame}


\begin{frame}[fragile]{Revisão}
  \centering
  \Large
  \begin{enumerate}
    \item Para que serve uma thread? Qual é a diferença em relação a um
      processo?
    \item O que é uma região crítica?
    \item Em que situação threads podem gerar inconsistências de dados?
    \item Como mutexes podem ajudar a resolver essas situações?
  \end{enumerate}
\end{frame}

\end{document}
