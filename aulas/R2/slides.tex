\documentclass{beamer}
%
% Choose how your presentation looks.
%
% For more themes, color themes and font themes, see:
% http://deic.uab.es/~iblanes/beamer_gallery/index_by_theme.html
%
\mode<presentation>
{
  \usetheme{Madrid}      % or try Darmstadt, Madrid, Warsaw, ...
  \usecolortheme{default} % or try albatross, beaver, crane, ...
  \usefonttheme{default}  % or try serif, structurebold, ...
  \setbeamertemplate{navigation symbols}{}
  \setbeamertemplate{caption}[numbered]
}

\usepackage[english]{babel}
\usepackage[utf8x]{inputenc}
\usepackage{graphicx}
\usepackage{array}

\title[Revisão 2]{EA879 -- Introdução ao Software Básico\\Revisão -- Parte 2}
\author{Tiago F. Tavares}
\institute{FEEC -- UNICAMP}
\date{Revisão}

\begin{document}

\begin{frame}
  \titlepage
\end{frame}

% Uncomment these lines for an automatically generated outline.
%\begin{frame}{Outline}
%  \tableofcontents
%\end{frame}

\section{Introdução}

\begin{frame}{Objetivos}
  \Large
  \begin{itemize}
    \item Identificar relações entre diferentes aspectos do conteúdo da aula
      até o momento
    \item Escrever textos concisos e efetivos que evidenciem essas relações
  \end{itemize}
\end{frame}

\begin{frame}[fragile]{Mapa conceitual}
  \centering
  \large
  Nossa disciplina começou com a análise de gramáticas regulares e culminou com
  a implementação de uma linguagem de domínio específico.
  \begin{enumerate}
    \item Junto à classe, vamos colocar na lousa o nome de todos os elementos
      que foram usados durante essa progressão (apenas substantivos!)
    \item <2-> Após, unieremos cada par de substantivos com um verbo, até
      completar o raciocínio todo.
  \end{enumerate}
\end{frame}

\begin{frame}[fragile]{Construção de frases}
  \centering
  \large
  \begin{enumerate}
    \item Escolha um par de substantivos que esteja ligado. Escreva uma frase que
  tenha o mesmo conteúdo que foi representado graficamente. A frase deve ter
      sentido. Corrija sua frase junto a seu grupo.
    \item<2-> Identifique o sujeito, o verbo e o objeto de sua frase.
    \item<3-> Escolha um substantivo que se ligue ao objeto de sua frase no mapa
      conceitual. Crie uma nova frase na qual o antigo objeto é o sujeito, o
      novo substantivo é o objeto e a frase contém o sentido deste novo nó.
    \item<4-> Pensando nas duas frases que você escreveu: é possível deixá-las
      menos repetitivas usando pronomes?
  \end{enumerate}
\end{frame}

\begin{frame}[fragile]{Reflexão}
  \centering
  \large
  Você assistirá a uma demonstração ao vivo da comparação de dois programas de
  multiplicação de matrizes, um escrito em C (uma linguagem compilada) e um em
  Python (uma linguagem interpretada). Faça predições (e explique as premissas
  que te levaram a pensar isso) para:
  \begin{itemize}
  \item Qual programa tem menos linhas de código?
  \item Qual programa é mais rápido?
  \item Qual programa levou mais tempo para implementar?
  \end{itemize}
  Após a demonstração, suas predições se confirmaram? O que isso quer dizer em
  relação a suas premissas?
\end{frame}

\begin{frame}[fragile]{Evidências}
  \centering
  \large
  Identifique o que são evidências e o que são conclusões
  \begin{enumerate}
  \item O código em Python tem 10 linhas.
  \item O código em C tem 133 linhas.
  \item O código em Python é mais simples que o código em C.
  \item O código em Python é mais rápido que o código em C.
  \item O código em Python executa 100 vezes mais rápido que o código em C.
  \end{enumerate}
  Após, escreva dois pequenos parágrafos conectando as evidências a suas
  respectivas conclusões.
\end{frame}

\begin{frame}[fragile]{Evidências}
  \centering
  O código em Python tem 10 linhas, ao passo que um programa equivalente em C
  tem 133 linhas. Portanto, o código em Python é mais simples que o código em C.
  \vspace{2cm}

  Em nossas medições, o programa escrito em Python executou 100 vezes mais
  rápido que o programa em C. Isso significa que o programa em Python é mais
  rápido.
\end{frame}

\begin{frame}[fragile]{Corrija as seguintes frases}
  \centering
  O código em Python tem 10 linhas. Um programa equivalente em C
  tem 133 linhas. O código em Python é mais simples que o código em C.
  \vspace{2cm}

  Em nossas medições, o programa escrito em Python executou 100 vezes mais
  rápido que o programa em C. Portanto, Python é uma linguagem mais rápida
  que C.
\end{frame}

\begin{frame}[fragile]{Encontre as hipérboles:}
  \centering
  \large
  Programas feitos em C são sempre avaliados como de extrema velocidade.
  Programas feitos em Python são constantemente tidos como muito lentos, embora
  tenham codificação muito simples e intuitiva.
\end{frame}

\begin{frame}[fragile]{Agora, encontre as especulações}
  \centering
  \large
  Programas feitos em C são \textbf{sempre} avaliados como de \textbf{extrema} velocidade.
  Programas feitos em Python são \textbf{constantemente} tidos como
  \textbf{muito} lentos, embora
  tenham codificação \textbf{muito simples e intuitiva}.
\end{frame}

\begin{frame}[fragile]{Há um problema...}
  \centering
  \large
  Programas feitos em C são \textit{ \textbf{sempre} avaliados como de
  \textbf{extrema} velocidade.}
  Programas feitos em Python são \textit{ \textbf{constantemente} tidos como
  \textbf{muito} lentos, embora
  tenham codificação \textbf{muito simples e intuitiva}.}
\end{frame}


\end{document}
