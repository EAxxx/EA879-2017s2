\documentclass{beamer}
%
% Choose how your presentation looks.
%
% For more themes, color themes and font themes, see:
% http://deic.uab.es/~iblanes/beamer_gallery/index_by_theme.html
%
\mode<presentation>
{
  \usetheme{Madrid}      % or try Darmstadt, Madrid, Warsaw, ...
  \usecolortheme{default} % or try albatross, beaver, crane, ...
  \usefonttheme{default}  % or try serif, structurebold, ...
  \setbeamertemplate{navigation symbols}{}
  \setbeamertemplate{caption}[numbered]
}

\usepackage[english]{babel}
\usepackage[utf8x]{inputenc}
\usepackage{graphicx}
\usepackage{array}

\title[08-Variáveis]{EA879 -- Introdução ao Software Básico\\Variáveis}
\author{Tiago F. Tavares}
\institute{FEEC -- UNICAMP}
\date{Aula 08 -- 15/agosto/2017}

\begin{document}

\begin{frame}
  \titlepage
\end{frame}

% Uncomment these lines for an automatically generated outline.
%\begin{frame}{Outline}
%  \tableofcontents
%\end{frame}

\section{Introdução}

\begin{frame}{Objetivos}
  \Large
  \begin{itemize}
    \item Entender como implementar variáveis em linguagens de programação
  \end{itemize}
\end{frame}


\begin{frame}[fragile]{Revisão}
  \centering
  \large
  Em C, quais das linhas abaixo contém operações de atribuição?
  \begin{verbatim}
    a = 5;
    a = (b==3);
    if (a==1) exit();
    int b;
    c = a + b + d;
  \end{verbatim}
\end{frame}

\begin{frame}[fragile]{Análise de código}
  \centering
  \large
  Junto ao seu grupo, analise o código da calculadora com variáveis. No código
  Yacc:
  \begin{enumerate}
    \item Que tokens estão associados à operação de atribuição?
    \item Quantas variáveis diferentes podem ser armazenadas nesta calculadora?
  \end{enumerate}

  No código Lex:
  \begin{enumerate}
    \item Que expressão regular é usada para reconhecer variáveis no código?
    \item Que operador é usado para representar uma operação de atribuição?
  \end{enumerate}
\end{frame}

\begin{frame}[fragile]{Análise de código}
  \centering
  \large
    Trace as árvores sintáticas, identificando as regras que foram ativadas, que
    permite resolver as expressões:
    \begin{verbatim}
    a = 5 + 6
    b = a + 12
    \end{verbatim}

\end{frame}

\begin{frame}[fragile]{Análise de código}
  \centering
  \large
    Uma operação de atribuição com auto-referência, ou seja:

    \begin{verbatim}
    a = a + 1
    \end{verbatim}

    deverá ser válida nesta linguagem?
\end{frame}

\begin{frame}[fragile]{Variando as variáveis}
  \centering
  \large
  As variáveis que temos para nossa calculadora são decritas por caracteres
  minúsculos. Identifique os problemas que poderemos encontrar por isso:
  \begin{enumerate}
    \item Os nomes das variáveis não correspondem semanticamente ao seu
      conteúdo.
    \item Não será possível fazer programas para nossa calculadora.
    \item Não será possível salvar nossas variáveis em disco.
    \item Talvez eu precise de mais variáveis que o sistema me disponibiliza.
  \end{enumerate}
\end{frame}

\begin{frame}{Problema: o arquivo de alunos}
\large
Um problema que temos hoje na disciplina é encontrar rapidamente dados (notas e
  avaliações) relacionados a algum aluno específico. Uma possível solução para
  isso é conseguir uma pasta de arquivos (daquelas verdes, com pastas supensas)
  e usá-la para guardar os dados todos.
  \begin{enumerate}
    \item Como cada pasta poderia ser identificada?
    \item Como as pastas poderiam estar organizadas entre si?
    \item O que deveria ser o conteúdo de cada pasta?
  \end{enumerate}
\end{frame}

\begin{frame}{Problema: o arquivo de alunos terceirizado}
\large
Decidimos então que nossos arquivos estão ficando muito abarrotados e ocupando
  as salas da FEEC que poderiam ser convertidas numa sala de vivência com café e
  sofá para todos. Por isso, terceirizamos a tarefa de guardar os dados. Porém,
  ainda precisamos acessá-los sob demanda. Então, contratamos a empresa
  GMF (Green Metal Files) para fazer a guarda dos arquivos.
  \begin{enumerate}
    \item Quando queremos acessar dados de um aluno, o que precisamos fornecer à
      GMF? O que ela deveria enviar como resposta ao nosso pedido?
    \item Quando queremos colocar um novo aluno, o que precisamos fornecer à
      GMF? O que ela deveria enviar como resposta ao nosso pedido?
  \end{enumerate}
\end{frame}

\begin{frame}{Metáfora Antropomórfica}
\Large
\centering
  É possível traçar um paralelo entre o arquivo de alunos e variáveis em um
  sistema. Nesse caso, o que o nome da variável representa? E o que o conteúdo
  da variável representa?
\end{frame}



\begin{frame}[fragile]{Cabeçalhos de funções de armazenamento}
\large
  Poderíamos definir funções para armazenar dados em um ``espaço de
  armazenamento'', de forma que pudéssemos definir os rótulos desses dados ao
  longo do programa. Nesse caso, precisaríamos de duas funções:

  \begin{verbatim}
  TIPO1 armazenar(ARGUMENTOS1);
  TIPO2 recuperar(ARGUMENTOS2);
  \end{verbatim}

  Defina o tipo e os argumentos de cada uma das funções.
\end{frame}


\end{document}
