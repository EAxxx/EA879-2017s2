\documentclass{beamer}
%
% Choose how your presentation looks.
%
% For more themes, color themes and font themes, see:
% http://deic.uab.es/~iblanes/beamer_gallery/index_by_theme.html
%
\mode<presentation>
{
  \usetheme{Madrid}      % or try Darmstadt, Madrid, Warsaw, ...
  \usecolortheme{default} % or try albatross, beaver, crane, ...
  \usefonttheme{default}  % or try serif, structurebold, ...
  \setbeamertemplate{navigation symbols}{}
  \setbeamertemplate{caption}[numbered]
}

\usepackage[english]{babel}
\usepackage[utf8x]{inputenc}
\usepackage{graphicx}
\usepackage{array}

\title[Revisão 1]{EA879 -- Introdução ao Software Básico\\Revisão -- Parte 1}
\author{Tiago F. Tavares}
\institute{FEEC -- UNICAMP}
\date{Revisão}

\begin{document}

\begin{frame}
  \titlepage
\end{frame}

% Uncomment these lines for an automatically generated outline.
%\begin{frame}{Outline}
%  \tableofcontents
%\end{frame}

\section{Introdução}

\begin{frame}{Objetivos}
  \Large
  \begin{itemize}
    \item Revisitar o conteúdo de aula até o momento
    \item Montar uma lista de exercícios sobre a disciplina
  \end{itemize}
  O material produzido hoje deverá ser entregue para que seja compilado e
  distribuído a todos os alunos.
\end{frame}

\begin{frame}[fragile]{Taxonomia de Bloom}
  \centering
  \large
  \begin{enumerate}
  \item Conhecer: sei que existe um interpretador.
  \item Entender: sei para que serve um interpretador.
  \item Aplicar: consigo usar um interpretador para executar um programa.
  \item Analisar: entendo as partes que compõem um interpretador.
  \item Avaliar: consigo dizer se um interpretador é bom ou ruim.
  \item Criar: consigo criar um novo interpretador.
  \end{enumerate}
\end{frame}


\begin{frame}[fragile]{Revisão}
  \centering
  \Large
  Junto ao seu grupo, faça uma lista de todos os tópicos que você aprendeu até o
  momento nesse curso. Use um nível de especificidade tão grande quando julgue
  necessário. Na sua lista, marque qual o nível da Taxonomia de Bloom
  que descreve seu nível de aprendizado em cada um dos tópicos.
\end{frame}

\begin{frame}[fragile]{Revisão}
  \centering
  \Large
  Junto ao seu grupo, faça uma lista de todos os tópicos que você aprendeu até o
  momento nesse curso. Na sua lista, marque qual o nível da Taxonomia de Bloom
  que descreve seu nível de aprendizado em cada um dos tópicos.
\end{frame}

\begin{frame}[fragile]{Revisão}
  \centering
  \large
  Discuta com seu grupo quais foram os três tópicos mais relevantes que você
  aprendeu durante esse curso (e marque quais são!). Após, pense numa atividade
  (ou numa pergunta) que só pode ser realizada por pessoas que dominam aquela
  atividade, no nível que você escolheu. Exemplos:
  \begin{itemize}
    \item ``entender um interpretador'': explique a diferença entre uma
  linguagem interpretada e uma linguagem compilada?
    \item ``criar um interpretador'': adicione operadores de exponenciação e de
      fatorial à calculadora que fizemos como exemplo.
  \end{itemize}
\end{frame}

\begin{frame}[fragile]{Revisão}
  \centering
  \Large
  Por fim, entregue a folha contendo as reflexões do grupo nesta atividade. As
  atividades e perguntas serão compiladas e distribuídas a todos.
\end{frame}



\end{document}
