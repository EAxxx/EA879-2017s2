\documentclass{beamer}
%
% Choose how your presentation looks.
%
% For more themes, color themes and font themes, see:
% http://deic.uab.es/~iblanes/beamer_gallery/index_by_theme.html
%
\mode<presentation>
{
  \usetheme{Madrid}      % or try Darmstadt, Madrid, Warsaw, ...
  \usecolortheme{default} % or try albatross, beaver, crane, ...
  \usefonttheme{default}  % or try serif, structurebold, ...
  \setbeamertemplate{navigation symbols}{}
  \setbeamertemplate{caption}[numbered]
}

\usepackage[english]{babel}
\usepackage[utf8x]{inputenc}
\usepackage{graphicx}
\usepackage{array}

\title[13-posix-e-arquivos]{EA879 -- Introdução ao Software
Básico\\Arquivos e POSIX}
\author{Tiago F. Tavares}
\institute{FEEC -- UNICAMP}
\date{Aula 13 -- 5/outubro/2017}

\begin{document}

\begin{frame}
  \titlepage
\end{frame}

% Uncomment these lines for an automatically generated outline.
%\begin{frame}{Outline}
%  \tableofcontents
%\end{frame}

\section{Introdução}

\begin{frame}{Objetivos}
  \Large
  \begin{itemize}
    \item Entender como arquivos operam em POSIX
    \item Entender a relação entre arquivos e ponteiros em POSIX
    \item Usar arquivos especiais STDIN e STDOUT
    \item Aplicar o redirecionamento de entradas e saídas (pipe)
  \end{itemize}
\end{frame}

\begin{frame}[fragile]{Previously on EA879...}
  \centering
  \Large
  Qual é o resultado de executar:
  \begin{verbatim}
  while(1) { fork(); }
  \end{verbatim}
  \begin{enumerate}
    \item Segmentation fault
    \item Programa trava no loop infinito
    \item Sistema operacional congela e o computador tem que ser resetado
    \item Nenhuma das anteriores
  \end{enumerate}
\end{frame}

\begin{frame}[fragile]{Arquivos em Linux}
  \centering
  \large
  \begin{enumerate}
    \item Entenda o código fonte que será mostrado
    \item <2-> Trace um mapa conceitual mostrando como um dado no disco rígido
      pôde ser acessado como uma variável na memória
  \end{enumerate}

\end{frame}

\begin{frame}[fragile]{Arquivos especiais: stdin e stdout}
  \centering
  \large
  Na demonstração de código a seguir, mantenha o foco em entender o que
  os identificadores stdin e stdout representam.
\end{frame}

\begin{frame}[fragile]{Redirecionamento de saída e de entrada}
  \centering
  \Large
  Analise o código do script de teste dos laboratórios e verifique o que os
  operadores < e > representam.
\end{frame}

\begin{frame}[fragile]{Processamento de arquivos com pouca memória}
  \centering
  \large
  Considere a sequência de instruções:
  \begin{itemize}
    \item Abrir arquivo (ele tem 50 TB)
    \item Carregar conteúdo do arquivo para a memória
    \item Somar 1 a todos os caracteres
    \item Abrir um novo arquivo
    \item Gravar o resultado no novo arquivo
    \item Fechar o novo arquivo
    \item Fechar o arquivo original
  \end{itemize}
  Modifique esse programa de forma que seja possível executá-lo usando somente
  1MB de memória RAM.
\end{frame}

\begin{frame}[fragile]{Processamento de arquivos com pouca memória}
  \centering
  \Large
  Qual dos programas começa a gerar saídas em menos tempo, à partir de sua
  execução?
\end{frame}

\begin{frame}[fragile]{Programas responsivos}
  \centering
  \Large
  Nos códigos-fonte de criptografia que analisaremos,
  verifique se o processamento está sendo
  feito em streaming ou se o arquivo é carregado inteiramente para a memória.

  Por que você acha que essa opção é vantajosa?
\end{frame}

\begin{frame}[fragile]{Análise}
  \centering
  \large
  Quais das situações abaixo poderiam descrever bem o nosso processo de
  programação até o momento:
  \begin{enumerate}
    \item Uma tarefa é dividida em partes iguais e direcionada para
      processos trabalhadores
    \item Um processo produz dados e um outro processa (consome) esses dados
    \item Para cada nova tarefa, um novo processo é gerado
  \end{enumerate}
\end{frame}

\begin{frame}[fragile]{Padrão de design produtor-consumidor}
  \centering
  \large
  Nas linhas de comando abaixo, identifique os produtores e os consumidores
  \begin{enumerate}
    \item ls $|$ grep c
    \item cat ``teste.txt'' $|$ wc
    \item cat ``teste.txt'' $|$ grep c $|$ wc
    \item <2-> uma fila de pipes: pipeline!
  \end{enumerate}

\end{frame}

\begin{frame}[fragile]{Demonstração}
  \centering
  \Large
  Verifique como invocar pipes de dentro de um programa escito em C. Qual é a
  vantagem de fazer isso?
\end{frame}

\begin{frame}[fragile]{Discussão}
  \centering
  \large
  Em quais situações abaixo poderíamos usar o padrão de design produtor-consumidor:
  \begin{enumerate}
    \item Gerar números aleatórios e então ordená-los
    \item Contar os caracteres de vários arquivos diferentes
    \item Processar um arquivo de texto em etapas pequenas, numa sequência
      pré-definida
    \item Simular uma série de pedais de efeito sonoro de guitarra (como plugins
      de áudio no Reaper)
    \item Manter uma interface gráfica rodando enquanto dados são processados
      num processo oculto ao usuário (como o MatLab)
    \item Dividir uma tarefa de busca entre vários trabalhadores diferentes
      (como o Google)
    \item Processar independentemente cada uma das requisições de cálculo que
      chega a um sistema (como o Wolfram Alpha)
  \end{enumerate}

\end{frame}




\end{document}
