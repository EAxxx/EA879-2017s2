\documentclass{beamer}
%
% Choose how your presentation looks.
%
% For more themes, color themes and font themes, see:
% http://deic.uab.es/~iblanes/beamer_gallery/index_by_theme.html
%
\mode<presentation>
{
  \usetheme{Madrid}      % or try Darmstadt, Madrid, Warsaw, ...
  \usecolortheme{default} % or try albatross, beaver, crane, ...
  \usefonttheme{default}  % or try serif, structurebold, ...
  \setbeamertemplate{navigation symbols}{}
  \setbeamertemplate{caption}[numbered]
}

\usepackage[english]{babel}
\usepackage[utf8x]{inputenc}
\usepackage{graphicx}
\usepackage{array}

\title[22-Tipagem]{EA879 -- Introdução ao Software
Básico\\Tipagem}
\author{Tiago F. Tavares}
\institute{FEEC -- UNICAMP}
\date{Aula 22 -- 16/novembro/2017}

\begin{document}

\begin{frame}
  \titlepage
\end{frame}

% Uncomment these lines for an automatically generated outline.
%\begin{frame}{Outline}
%  \tableofcontents
%\end{frame}

\section{Introdução}

\begin{frame}[fragile]{Previously, on EA879...}
  \centering
  \Large
  \begin{itemize}
    \item Compiladores: RegEx e GLC
    \item Processos, threads, preempção e mutexes
    \item Pilha e Heap
  \end{itemize}
\end{frame}

\begin{frame}{Objetivos}
  \Large
  \begin{itemize}
    \item Entender o que é tipagem estática e o que é tipagem dinâmica
    \item Entender as vantagens e overheads de usar tipagem dinâmica
  \end{itemize}
\end{frame}



\begin{frame}[fragile]{Exercício 1}
  \centering
  \Large
\end{frame}

\begin{frame}[fragile]{Exercício 2}
  \centering
  \Large
  \begin{enumerate}
    \item <2-> Conteúdo da variável e tipo da variável
    \item <3-> \textsc{malloc()} e \textsc{free()}
    \item <4-> Código
    \item <5-> São criados com \textsc{malloc()}, portanto aparecem no heap.
  \end{enumerate}
\end{frame}

\begin{frame}[fragile]{Exercício 3}
  \centering
  \Large
  Exercício 3
  \begin{enumerate}
    \item <2-> 3, 4, 5, 2, 1
    \item <3-> O espaço para dados já está alocado na pilha, portanto só é
      preciso copiar os dados.
  \end{enumerate}
\end{frame}

\begin{frame}[fragile]{Exercício 4}
  \centering
  \Large
  Exercício 4
  \begin{enumerate}
    \item <2-> 1
    \item <3-> 2
    \item <4-> 3
    \item <5-> 2
    \item <6-> 1
    \item <7-> 0
    \item <8-> Desalocar: após a linha \textsc{B=10}.
    \item <9-> Quando a contagem de referências vai a zero.
  \end{enumerate}
\end{frame}

\begin{frame}[fragile]{Exercício 5}
  \centering
  \Large
  Exercício 5
  \begin{verbatim}
  A = [1 2 3];
  B = A;
  B(1) = 50;
  A;
  \end{verbatim}
\end{frame}

\begin{frame}[fragile]{Exercício 6}
  \centering
  \Large
  Exercício 6
\end{frame}

\begin{frame}[fragile]{Exercício 7}
  \centering
  \Large
  Exercício 7
\end{frame}


\begin{frame}[fragile]{Trabalho 2}
  \centering
  \Large
  \begin{itemize}
  \item <2-> Webserver: I/O bounded!
  \item <3-> Banco de dados: I/O bounded!
  \item <4-> Jogo de computador: CPU bounded!
  \item <5-> Processador de vídeos: CPU bounded!
  \end{itemize}
\end{frame}

\end{document}

