\documentclass{beamer}
%
% Choose how your presentation looks.
%
% For more themes, color themes and font themes, see:
% http://deic.uab.es/~iblanes/beamer_gallery/index_by_theme.html
%
\mode<presentation>
{
  \usetheme{Madrid}      % or try Darmstadt, Madrid, Warsaw, ...
  \usecolortheme{default} % or try albatross, beaver, crane, ...
  \usefonttheme{default}  % or try serif, structurebold, ...
  \setbeamertemplate{navigation symbols}{}
  \setbeamertemplate{caption}[numbered]
}

\usepackage[english]{babel}
\usepackage[utf8x]{inputenc}
\usepackage{graphicx}
\usepackage{array}

\title[09-multitasking-tempo-real]{EA879 -- Introdução ao Software
Básico\\Multitasking em tempo real}
\author{Tiago F. Tavares}
\institute{FEEC -- UNICAMP}
\date{Aula 09 -- 15/agosto/2017}

\begin{document}

\begin{frame}
  \titlepage
\end{frame}

% Uncomment these lines for an automatically generated outline.
%\begin{frame}{Outline}
%  \tableofcontents
%\end{frame}

\section{Introdução}

\begin{frame}{Objetivos}
  \Large
  \begin{itemize}
    \item Entender como
      sistemas podem parecer executar várias tarefas ao mesmo tempo.
    \item Enteder as restrições que o aspecto multitarefa implica para os
      programadores.
  \end{itemize}
\end{frame}


\begin{frame}[fragile]{Revisão: interrupções}
  \centering
  \large
  Ordene os eventos que acontecem durante uma chamada de interrupção externa num
  microcontrolador:
  \begin{enumerate}
    \item Dispositivo eletrônico emite borda de subida num pino do MCU.
    \item MCU interrompe o que está executando.
    \item MCU retoma o que estava executando.
    \item Flag de interrupção é ativada.
    \item Flag de \textit{enable} na interrupção é ativada.
    \item Endereço da rotina de interrupção é colocada no vetor de interrupções.
    \item MCU executa chamada de função marcada no vetor de interrupções.
    \item Rotina de interrupção encerra.
    \item Evento físico acontece.
  \end{enumerate}
\end{frame}

\begin{frame}[fragile]{Análise de código}
  \centering
  \large
  Quais linhas abaixo travam o MCU num laço?
  \begin{verbatim}
  int main() {
    while (1) {
      acende_led(1); /* Acende o LED 1 */
      wait(500); /* espera 500ms */
      apaga_led(1);
      wait(500);
    }
  }
  \end{verbatim}
\end{frame}

\begin{frame}[fragile]{Análise de código}
  \centering
  \large
  Modifique o código abaixo para que o LED 2 pisque com o dobro do período do
  LED 1. Após, modifique-o novamnete de tal forma que o LED 3 pisque com dez
  vezes o período do LED 1.
  \begin{verbatim}
  int main() {
    while (1) {
      acende_led(1);
      wait(500); /* espera 500ms */
      apaga_led(1);
      wait(500);
    }
  }
  \end{verbatim}
\end{frame}

\begin{frame}[fragile]{Pão com manteiga na chapa e limonada}
  \centering
  \large
  Mostre, através de um diagrama, um processo que permita a uma pessoa espremer
  50 limões e fazer 50 pães com manteiga na chapa sem que ela perca tempo
  esperando o pão com manteiga ficar pronto e sem que o pão com manteiga queima,
  sabendo que o tempo que leva entre um pão ficar pronto e ele queimar é menor
  que o tempo que leva para espremer um limão. Que tipos de alarmes são
  necessários para que isso seja possível?
\end{frame}

\begin{frame}[fragile]{Um pequeno kernel de tempo real}
  \centering
  \large
  Analise o código do kernel de tempo real. Identifique:
  \begin{enumerate}
    \item Quais variáveis são responsáveis pelo controle de fluxo na função main()
    \item Como essas variáveis são modificadas
    \item Qual é e qual evento ativa a função de interrupção
    \item De quanto em quanto tempo cada uma das funções auxiliares é chamada
    \item Como é possível alterar o período de chamadas de cada uma das funções
      auxiliares
    \item Como é possível adicionar uma nova função auxiliar periódica nesta
      estrutura
  \end{enumerate}
\end{frame}

\begin{frame}[fragile]{Operando com o kernel}
  \centering
  \large
  Modifique o código do kernel para que ele possa fazer piscarem três LEDS, com
  períodos 1s, 3s e 10s. Como esse código se compara com o que foi proposto
  anteriormente?
\end{frame}

\begin{frame}[fragile]{Restrições do RT-Kernel}
  \centering
  \large
  Qual seria a consequência das seguintes eventualidades em funções periódicas
  num programa baseado na  estrutura de kernel de tempo real:
  \begin{enumerate}
  \item Alterar uma variável global
  \item Ter período igual ao relacionado a outra função
  \item Levar mais tempo que seu próprio período
  \item Levar mais tempo que o período da rotina de interrupção
  \item Entrar em loop infinito
  \end{enumerate}

\end{frame}


\end{document}
