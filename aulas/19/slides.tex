\documentclass{beamer}
%
% Choose how your presentation looks.
%
% For more themes, color themes and font themes, see:
% http://deic.uab.es/~iblanes/beamer_gallery/index_by_theme.html
%
\mode<presentation>
{
  \usetheme{Madrid}      % or try Darmstadt, Madrid, Warsaw, ...
  \usecolortheme{default} % or try albatross, beaver, crane, ...
  \usefonttheme{default}  % or try serif, structurebold, ...
  \setbeamertemplate{navigation symbols}{}
  \setbeamertemplate{caption}[numbered]
}

\usepackage[english]{babel}
\usepackage[utf8x]{inputenc}
\usepackage{graphicx}
\usepackage{array}

\title[19-Tempo]{EA879 -- Introdução ao Software
Básico\\Tempo de execução}
\author{Tiago F. Tavares}
\institute{FEEC -- UNICAMP}
\date{Aula 19 -- 09/outubro/2017}

\begin{document}

\begin{frame}
  \titlepage
\end{frame}

% Uncomment these lines for an automatically generated outline.
%\begin{frame}{Outline}
%  \tableofcontents
%\end{frame}

\section{Introdução}

\begin{frame}{Objetivos}
  \Large
  \begin{itemize}
    \item Entender a diferença entre real time, user time e system time
    \item Entender como medir tempo de execução de funções individuais
    \item Diferenciar CPU-bound de I/O bounded
  \end{itemize}
\end{frame}

\begin{frame}[fragile]{Previously, on EA879...}
  \centering
  \Large
  \begin{itemize}
    \item Compiladores: RegEx e GLC
    \item Processos, threads, preempção e mutexes
    \item Pilha e Heap
  \end{itemize}
\end{frame}

\begin{frame}[fragile]{Exercício 1}
  \centering
  \Large
  \begin{enumerate}
    \item <2-> Executando instruções: micro-segundos
    \item <3-> Tempo total: por volta de 1 segundo
    \item <4-> (a) = user; (b) = real
  \end{enumerate}
\end{frame}

\begin{frame}[fragile]{Exercício 2}
  \centering
  \Large
  Exercício 2
  \begin{enumerate}
    \item <2-> User time: \textsc{clock()}
    \item <3-> Real time: \textsc{gettimeofday()}
  \end{enumerate}
\end{frame}

\begin{frame}[fragile]{Exercício 3}
  \centering
  \Large
  Exercício 3
  \begin{enumerate}
    \item <2-> \textsc{fopen()}
    \item <3-> \textsc{fgetc()}
    \item <4-> \textsc{fclose()}
    \item <5-> Executar \textsc{time}
  \end{enumerate}
\end{frame}

\begin{frame}[fragile]{Exercício 4}
  \centering
  \Large
  Exercício 4
  \begin{enumerate}
    \item <2-> O que é mais rápido: alocar memória na pilha ou no heap?
  \end{enumerate}
\end{frame}

\begin{frame}[fragile]{Exercício 5}
  \centering
  \Large
  Exercício 5
  \begin{enumerate}
    \item <2-> \textsc{fibo\_rec()}
    \item <3-> \textsc{void *fibo(void *args)}
    \item <4-> \textsc{void *fibo\_threads(void *args)}: cada thread calcula
      metade da ``árvore''
    \item <5-> Fazer hipótese. Executar \textsc{time}
  \end{enumerate}
\end{frame}

\begin{frame}[fragile]{Exercício 6}
  \centering
  \Large
  Exercício 6
\end{frame}

\begin{frame}[fragile]{Exercício 7}
  \centering
  \Large
  Exercício 7
\end{frame}


\begin{frame}[fragile]{Trabalho 2}
  \centering
  \LARGE
  Medição de tempo de execução (real time!)
\end{frame}



\end{document}
