\documentclass{beamer}
%
% Choose how your presentation looks.
%
% For more themes, color themes and font themes, see:
% http://deic.uab.es/~iblanes/beamer_gallery/index_by_theme.html
%
\mode<presentation>
{
  \usetheme{Madrid}      % or try Darmstadt, Madrid, Warsaw, ...
  \usecolortheme{default} % or try albatross, beaver, crane, ...
  \usefonttheme{default}  % or try serif, structurebold, ...
  \setbeamertemplate{navigation symbols}{}
  \setbeamertemplate{caption}[numbered]
}

\usepackage[english]{babel}
\usepackage[utf8x]{inputenc}
\usepackage{graphicx}
\usepackage{array}

\title[23-Linking]{EA879 -- Introdução ao Software
Básico\\Bibliotecas estáticas e dinâmicas}
\author{Tiago F. Tavares}
\institute{FEEC -- UNICAMP}
\date{Aula 23 -- 23/novembro/2017}

\begin{document}

\begin{frame}
  \titlepage
\end{frame}

% Uncomment these lines for an automatically generated outline.
%\begin{frame}{Outline}
%  \tableofcontents
%\end{frame}

\section{Introdução}

\begin{frame}[fragile]{Previously, on EA879...}
  \centering
  \Large
  \begin{itemize}
    \item Compiladores: RegEx e GLC
    \item Processos, threads, preempção e mutexes
    \item Pilha e Heap
    \item Linguagens com tipagem estática e tipagem dinâmica
  \end{itemize}
\end{frame}

\begin{frame}{Objetivos}
  \Large
  \begin{itemize}
    \item Entender a diferença entre interface e implementação
    \item Entender como distribuir e compartilhar código
  \end{itemize}
\end{frame}



\begin{frame}[fragile]{Exercício 1}
  \centering
  \Large
  \begin{enumerate}
    \item <2-> 01
    \item <3-> 02
    \item <4->O código usado para a implementação da função \textsc{imprimir()}
      mudou.
    \item <5-> O cabeçalho da função deve ser o mesmo
    \item <6-> Definida: \textsc{interface.h}. Implementada:
      \textsc{interface.c}.
  \end{enumerate}
\end{frame}

\begin{frame}[fragile]{Exercício 2}
  \centering
  \Large
  \begin{enumerate}
    \item <2-> Inicializa um vetor, calcula a média dos valores do vetor,
      calcula a variância dos valores do vetor.
    \item <3-> Ponteiro para o vetor e tamanho do vetor.
    \item <4-> Retornam a média e a variância do vetor.
    \item <5-> Quais aspectos do código facilitam sua compreensão??
  \end{enumerate}
\end{frame}

\begin{frame}[fragile]{Exercício 3}
  \centering
  \Large
  Exercício 3
  \begin{enumerate}
    \item <2-> Sim, poderia, uma vez que os dados passados nos cabeçalhos e os
      tipos de retorno são os mesmos.
    \item <3-> Não há consistência da ordem dos elementos nos cabeçalhos. Não há
      correspondência semântica entre os rótulos das funções e variáveis e suas
      aplicações.
  \end{enumerate}
\end{frame}

\begin{frame}[fragile]{Exercício 4}
  \centering
  \Large
  Exercício 4
  \begin{enumerate}
    \item <2-> No Heap, já que a memória é alocada dinamicamente.
    \item <3-> Ela recebe como parâmetro um ponteiro para o retângulo do
      qual deve calcular a área.
    \item <4-> Um ponteiro para qualquer tipo de dado tem o mesmo número de
      bytes, então o typecasting é válido.
  \end{enumerate}
\end{frame}

\begin{frame}[fragile]{Exercício 5}
  \centering
  \Large
  \begin{enumerate}
  \item <2-> retangulos.h deve conter os cabeçalhos das funções.
  \item <3-> Os cabeçalhos não têm referência para a struct retangulo, então não
    é necessário defini-la no header file.
  \item <4-> Demonstração!
  \end{enumerate}
\end{frame}

\begin{frame}[fragile]{Exercício 6}
  \centering
  \Large
  Exercício 6
  \begin{enumerate}
    \item <2-> Dinâmica
    \item <3-> Estática
    \item <4-> Estática
    \item <5-> Ambas
  \end{enumerate}
\end{frame}

\begin{frame}[fragile]{Exercício 7}
  \centering
  \Large
  Exercício 7
  \begin{enumerate}
    \item <2-> Demonstração da interface para ser modificada: código-fonte do
      programa de interface
    \item <3-> O algoritmo é patenteado: header file + código compilado
    \item <4-> Oferecer atualizações: biblioteca dinâmica
    \item <5-> \textsc{main.c}, \textsc{retangulos.h}, \textsc{retangulos.so}
  \end{enumerate}
\end{frame}

\end{document}

