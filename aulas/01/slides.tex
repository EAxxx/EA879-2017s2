\documentclass{beamer}
%
% Choose how your presentation looks.
%
% For more themes, color themes and font themes, see:
% http://deic.uab.es/~iblanes/beamer_gallery/index_by_theme.html
%
\mode<presentation>
{
  \usetheme{Madrid}      % or try Darmstadt, Madrid, Warsaw, ...
  \usecolortheme{default} % or try albatross, beaver, crane, ...
  \usefonttheme{default}  % or try serif, structurebold, ...
  \setbeamertemplate{navigation symbols}{}
  \setbeamertemplate{caption}[numbered]
}

\usepackage[english]{babel}
\usepackage[utf8x]{inputenc}
\usepackage{graphicx}
\graphicspath{{./fig/}}

\title[01-Introdução]{EA879 -- Introdução ao Software Básico\\Plano de ação}
\author{Tiago F. Tavares}
\institute{FEEC -- UNICAMP}
\date{Aula 01 -- 8/agosto/2017}

\begin{document}

\begin{frame}
  \titlepage
\end{frame}

% Uncomment these lines for an automatically generated outline.
%\begin{frame}{Outline}
%  \tableofcontents
%\end{frame}

\section{Introdução}

\begin{frame}{Equipe}

\begin{itemize}
  \item Tiago F. Tavares -- sala 311 -- \url{tavares@dca.fee.unicamp.br}
  \item Raul Ícaro -- \url{raulicaro@icloud.com}
\end{itemize}

\vskip 1cm

\begin{block}{Como entrar em contato}
Mande um e-mail, use o Moodle ou bata na porta. Geralmente, respondo e-mails em menos de 48h.
\end{block}
\end{frame}


\begin{frame}{Site do curso}
\LARGE
Para coordenar nosso curso, utilizaremos o Moodle. Já há atividades no Moodle a serem realizadas, então confira se o seu acesso está liberado.
\end{frame}

\begin{frame}{Horário}
\LARGE
\begin{itemize}
\item Início: 16h00
\item Final: 17h45 ($\pm 15$ minutos)
\item Sala: PE-12 (prédio da pós-graduação)
\end{itemize}
\end{frame}

\begin{frame}{Objetivo do curso}
\LARGE
Ao fim deste curso, seremos capazes de usar ferramentas relacionadas a compiladores e sistemas operacionais para \textbf{otimizar programas de computador} e \textbf{trabalhar em equipe} de forma eficaz.
\end{frame}

\begin{frame}{Ementa}
Ementa DAC: Revisão sobre linguagem assembly, linguagem C e montadores. Estruturas de dados. Compiladores. Sistemas Operacionais.

\vskip 1cm

\begin{block}{Em outras palavras:}
\begin{itemize}
\item Por que existem linguagens de programação?
\item Como podemos fazê-las efetivas para resolver problemas?
\item Como é possível coordenar o trabalho de muitas pessoas em um único sistema de computação?
\end{itemize}
\end{block}
\end{frame}

\begin{frame}{Referências bibliográficas}
Nesta disciplina, adotaremos uma postura científica e crítica quanto a todo o nosso aprendizado. Isso significa que nada deve ser tomado como verdade, a não ser que seja construído à partir de evidências reais. Isso significa que ``estar no livro'' não é um argumento válido para algo ser tomado como verdade. Mesmo assim, livros e outras referências são úteis para guiar a própria busca por evidências. Algumas referências:
\begin{enumerate}
\item Ricarte, I., ``Introdução à Compilação''. Ed. Elsevier.
\item Tannenbaum, A. S., ``Sistemas Operacionais Modernos''. Ed. Pearson.
\item Bibliografia selecionada relacionada a cada assunto.
\end{enumerate}
\end{frame}


\begin{frame}{Atividades extra-sala}
\Large
Neste curso, teremos atividades extra-sala semanalmente. Elas envolverão:
\begin{enumerate}
\item \textbf{Atividades de programação}, previstas para tomar, em média, menos de 2h.
\item \textbf{Atividades tipo questionátio Moodle}, previstos para tomar 15 minutos.
\item \textbf{Trabalhos em grupo} (previsão: 3 trabalhos).
\end{enumerate}
\end{frame}


\begin{frame}{Atividades de programação}
\Large
\begin{itemize}
\item Atividades INDIVIDUAIS.
\item Pequenos problemas de computação envolvendo os conceitos abordados nas aulas anteriores.
\item Provavelmente chamarão a atenção para o problema que será abordado na semana seguinte à entrega.
\item Entrega via Github (veja instruções no site da disciplina).
\item Avaliação: testes de entrada-saída.
\end{itemize}
\end{frame}

\begin{frame}{Atividades de questionário Moodle}
\Large
\begin{itemize}
\item Atividades INDIVIDUAIS.
\item Perguntas conceituais, múltipla escolha, sobre os conceitos abordados nas aulas anteriores.
\item Provavelmente envolverão a síntese de conceitos relacionados a diversas aulas anteriores.
\item Entrega via Moodle.
\item Avaliação: erro/acerto.
\item Em caso de erro, uma nova tentativa pode ser feita, valendo 75\% da nota total.
\end{itemize}
\end{frame}

\begin{frame}{Trabalhos em grupo}
\Large
\begin{itemize}
\item Atividades EM GRUPO (3-5 pessoas).
\item Envolverá programação e elaboração de um relatório \textbf{curto}.
\item Teremos aulas específicas sobre como elaborar um relatório curto e eficaz.
\item Entrega via Moodle.
\item Avaliação: rubrica, chamando atenção especial à análise de resultados.
\end{itemize}
\end{frame}

\begin{frame}{Atividades em sala}
\Large
Quando estivermos em sala de aula, trabalharemos sempre \textbf{em grupo} (3-5 pessoas). Assim:
\begin{enumerate}
\item Ao chegar na sala de aula, posicione as cadeiras de forma a facilitar o trabalho do grupo.
\item Ao sair da sala de aula, retorne as cadeiras para a posição original. Sempre devolveremos a sala à FEEC num estado mais limpo do que como a encontramos.
\end{enumerate}
\end{frame}

\begin{frame}{Ensino ativo}
\Large
Nesta disciplina, utilizaremos técnicas de ensino ativo inspiradas em Problem-Based Learning (PBL), adaptadas às condições materiais que temos atualmente.
\end{frame}

\begin{frame}{}
\includegraphics[width=\textwidth]{deathbypowerpoint.png}
\end{frame}

\begin{frame}{Formar grupos}
\Large
Forme um grupo de 3 a 5 pessoas (nem mais e nem menos). Se necessário, apresentem-se (qual seu nome? Que curso você faz? Qual é seu hobby?).

Descubra qual é o membro do grupo que assistiu a mais séries no Netflix nas férias de julho.
\end{frame}

\begin{frame}{Processos de aprendizado}
\Large
\begin{enumerate}
\item Individualmente, pense em algo que você aprendeu na sua vida e que foi muito representativo para você. Encontre os seguintes elementos no seu processo:
\begin{enumerate}
\item O que você aprendeu?
\item Por que você pode dizer, hoje, que aprendeu?
\item Por que acredita que esse processo foi representativo?
\end{enumerate}
\item<2->Compartilhe suas reflexões com seu grupo. Quais elementos existem em comum?
\end{enumerate}
\end{frame}

\begin{frame}{Problema 1: grupos}
\large
Gostaríamos que todos os grupos fossem plenamente funcionais. Para isso, discutiremos as condições que precisamos/podemos criar para que os grupos funcionem bem. Assim, em grupo, discuta:
\begin{enumerate}
\item Quais são as vantagens de trabalhar em grupo em relação a trabalhar individualmente? Quais são as desvantagens?
\item O que significa um grupo de estudos funcionar bem? O que significa um grupo de estudos funcionar mal?
\item<2-> Quais são características que podem ser observadas em um grupo que funciona bem?
\item<2-> Quais são características que podem ser observadas em um grupo que funciona mal?
\item<3-> Partindo de exemplos da vida de cada um dos membros do grupo, encontre condições que favorecem o bom e o mau funcionamento de grupos de estudo. Como podemos criar as condições benéficas e evitar as condições maléficas?
\end{enumerate}
\end{frame}

\begin{frame}{Problema 2: ativididades do curso}
\large
Atualmente, sabemos que teremos as seguintes atividades no curso:
\begin{enumerate}
\item Atividades de programação individual (A),
\item Atividades tipo questionário Moodle (B),
\item Trabalhos extra-sala em grupo (C),
\item Participação em sala (D).
\end{enumerate}

Discuta com seu grupo se todos estão confortáveis com essas atividades, ou se o grupo gostaria de propor outras atividades (apresentação estilo seminário? provas? vídeos? apresentação de teatro? alguma outra coisa?) para o curso.
\end{frame}

\begin{frame}{Problema 3: notas}
\large
Atualmente, sabemos que teremos as seguintes atividades com nota no curso:
\begin{enumerate}
\item Atividades de programação individual (A),
\item Atividades tipo questionário Moodle (B),
\item Trabalhos extra-sala em grupo (C),
\item (outras atividades propostas pelos grupos?)
\end{enumerate}
Discuta com seu grupo uma proposta para equacionar as notas das atividades gerando uma nota final, de tal forma que a nota seja uma medida honesta do aprendizado do aluno, que isso não gere stress desenecessário aos alunos e, que, ao mesmo tempo, não permita que o curso seja abandonado ao longo do semestre.
\end{frame}




\end{document}
